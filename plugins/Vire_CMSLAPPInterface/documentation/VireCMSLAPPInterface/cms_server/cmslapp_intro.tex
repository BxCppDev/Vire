%%%%%%%%%%%%%%%%%%%%%%%%%
\subsection{Introduction}

The CMS/LAPP  server is the main  service that allows the  Vire server
(and  possibly its  Vire  clients) to  access  the hardware  resources
managed     by     MOS     servers      in     the     OPCUA     space
(Fig. \ref{fig:cms_server:layout_0a}).

\begin{figure}[h]
\begin{center}
\scalebox{0.75}{\input{\pdftextimgpath/cms_server-layout_0a.pdftex_t}}
\end{center}
\caption{Overview of  the Vire server,  one Vire client,  the CMS/LAPP
  server and  its connections  to the  hardware through  dedicated MOS
  servers.}
\label{fig:cms_server:layout_0a}
\end{figure}

\noindent From the Vire server point of view, the CMS/LAPP server is a
subcontractor agent which behaves like  a \emph{proxy}, hidding the
details  of  the  real  time   and  low  level  communication  systems
implemented in  MOS servers  and their  embedded device  drivers.  Its
role is to provide, in  terms of Vire \emph{resources}, a conventional
access  to  the  datapoints   (DP)  and  dedicated  actions  (methods)
associated to MOS devices.

\vfill
\afterpage{\clearpage}
\pagebreak
