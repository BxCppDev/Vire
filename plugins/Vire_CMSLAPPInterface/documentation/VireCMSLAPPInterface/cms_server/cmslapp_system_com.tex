
\subsection{System communication between the Vire server and the CMS/LAPP server}

%% \Large
%% \textcolor{red}{2017-05-03 FM: This section is obsolete and must be rewritten.}
%% \normalsize

This section  describes the  specific \emph{channels}  implemented for
\textcolor{blue}{service} messaging  between the  Vire server  and the
CMS/LAPP server.

\subsubsection{Vire server to CMS/LAPP server service channel (RPC)}

The architecture  of the CMS  system implies  that the Vire  server is
basically a client with respect to the CMS/LAPP server.

The  Vire API  defines  the concept  of  \emph{communication plug}  to
represent a kind of network  operation. There are four different types
of  \emph{communication plug}.   To  implement  the RPC  (\emph{Remote
  Procedure Call})  layer with  the CMS/LAPP  server, the  Vire server
uses a \emph{service} (or \emph{RPC}) client plug.
A RPC client plug is conventionnaly
configured to send some service RPC request message from the Vire server to a
dedicated  private queue  named  \texttt{"subcontractor.service"}.   The
CMS/LAPP server then extracts and  decodes the  request message  from this
queue, performs some actions then sends a response message back to the
Vire server through a dedicated private queue where the \emph{RPC client plug}
reads      the      response.       The name of this      queue
is  arbitrary, it is transmitted via the \texttt{reply\_to} AMQP message property
associated to the original request message.  Practically,  the Vire server
will decide to create such \emph{response} queues depending on the needs.
Technically, the  CMS/LAPP server  does not  need to  know the  name of
this queue because  each request  message contains  this identifier, through the
RabbitMQ \texttt{"reply\_to"} AMQP message property.

The service  RPC is  always used synchronously.   The Vire  server RPC
client plug operates in blocking mode.  When it receives a RPC request
message, the  CMS/LAPP server must  send back the  associated response
message and it is not allowed  to use the \emph{"reply\_to"} queue to send
some other arbitrary  messages.  To ensure some limited  control of the
RPC  process,  the request  message  sent  by  the RPC client  plug  is
conventionnaly  given  a   \texttt{"correlation\_id"}  property.   The
CMS/LAPP  server must  set  the  same \texttt{"correlation\_id"}  AMQP
property in  the response  message.  It is  the responsability  of the
Vire server to choose the  value of this \emph{correlation identifier}
(a character string) using an arbitrary internal policy.

\begin{figure}[h]
\begin{center}
\scalebox{0.9}{\input{\pdftextimgpath/vire_cmslapp_rpc_arch_0.pdftex_t}}
\end{center}
\caption{The RPC  channel for  service communication between  the Vire
  server (request  emitter/response receiver) and the  CMS/LAPP server
  (request receiver/response emitter).}\label{fig:cmslapp_rpc:arch_0}
\end{figure}


\vfill
\afterpage{\clearpage}
\pagebreak

%%%%%%%%%%%%%%%%%%%%%%%%%%%%%%%%%%%%%%%%%%%%%%%%%%%%%%%%%%%%%%%%%%%%%%%%%%%%%%%%%%
\subsubsection{General description of RPC messaging between the Vire server and the CMS/LAPP server}

Let's now  consider the following use  case: the Vire server  sends an
RPC request to  the CMS/LAPP server through the RPC  channel. How does
it works  practically?  In the  following, we illustrate  the detailed
sequence  of   operations  needed  to  fulfill   the  synchronous  RPC
transaction:

\begin{enumerate}

\item The Vire  server instantiates a payload object  which represents a
  specific action to be performed by the CMS/LAPP server (execute a method, set a value,
  activate some functionality\dots):
  \vskip 10pt
  \begin{center}
    \scalebox{0.5}{\input{\pdftextimgpath/vire_cmslapp_rpc_payload_1.pdftex_t}}
  \end{center}

  \noindent Within the  Vire system, only three  categories of payload
  exist:

  \begin{itemize}

  \item \emph{request} :  corresponds to RPC request sent  by the Vire
    server to the CSM/LAPP server.  They are always sent synchronously
    using the RPC mechanism using a mailbox of type \emph{service}.

  \item \emph{response} : corresponds to RPC response sent back by the
    CSM/LAPP  server  to  the  Vire   server.  They  are  always  sent
    synchronously using the  RPC mechanism and are  correlated to an
    unique RPC request. The original emitter of the RPC request must setup a private
    queue to receive a RPC response.

  \item  \emph{event}  :  corresponds  to  some  data  set/information
    (alarm/signal/log entry, asynchronous RPC response event)  which is  sent asynchronously
    from one software component to another (or more) using a mailbox of type \emph{event}..

  \end{itemize}

  \noindent In the present case, the  payload object built by the Vire
  server belongs to the \emph{request} category.

\item  The Vire  server submits  the  payload object  to the  \emph{RPC
  client       plug}      which       is      binded       to      the
  \texttt{"subcontractor.service"}       queue      in       the\\
  \textcolor{blue}{\texttt{/supernemo/demonstrator/cms/vire/subcontractors/cmslapp/system}}     virtual
  host.
  \vskip 10pt
  \begin{center}
    \scalebox{0.5}{\input{\pdftextimgpath/vire_cmslapp_rpc_payload_2.pdftex_t}}
  \end{center}


\item The \emph{RPC client plug} builds a message object which contains:

  \begin{itemize}

  \item a header with mandatory attributes and possibly some arbitrary
    metadata than will be used by the recipient, i.e. the CMS/LAPP
    server.   The layout  of  the message  header  is  described
    elsewhere  in this  document. It  is  expected that parts  of
    metadata stored in the header will  be used by the CMS/LAPP server
    to accomplish the RPC transaction.

  \item  a body  which contains  the  type identifier  of the  request
    payload object and the payload object itself.

  \end{itemize}
  \vskip 10pt
  \begin{center}
    \scalebox{0.7}{\input{\pdftextimgpath/vire_cmslapp_rpc_payload_3.pdftex_t}}
  \end{center}


\item The  \emph{RPC client plug} encode/serialize  the message object
  in a buffer using some encoding technique (Protobuf\dots):
  \vskip 10pt
  \begin{center}
    \scalebox{0.7}{\input{\pdftextimgpath/vire_cmslapp_rpc_payload_4.pdftex_t}}
  \end{center}

\item The \emph{RPC  client plug} wraps the buffer  in a AMQP/RabbitMQ
  message  (the   envelope)  and   associates to it a set of special   AMQP  message
  properties:

  \begin{itemize}

  \item    \texttt{"content\_type"}    property:   the encoding format used for message.\\
    Example:  \texttt{"application/x-protobuf"} (for    Protobuf
    serialization).

  \item \texttt{"correlation\_id"} property:  the \emph{RPC client plug}
    generates an unique character string for this request payload. \\
    Example:
    \texttt{"correlation\_id" = "snemo.demonstrator.cms.calo.hv.56327"}.

  \item \texttt{"reply\_to"} property:  the \emph{RPC client plug}
    indicated the name of the queue where to CMS/LAPP server must publish
    the response message. \\
    Example:
    \texttt{"reply\_to" = "vireserver.service.response.uFaezae2"}.

  \end{itemize}
  \vskip 10pt
  \begin{center}
    \scalebox{0.7}{\input{\pdftextimgpath/vire_cmslapp_rpc_payload_5.pdftex_t}}
  \end{center}

\item  The \emph{RPC  client  plug}  publishes the RabbitMQ
  message to the \texttt{"subcontractor.service"} queue:
  \vskip 10pt
  \begin{center}
    \scalebox{0.6}{\input{\pdftextimgpath/vire_cmslapp_rpc_payload_6.pdftex_t}}
  \end{center}
  and blocks until it receives the response message from the CMS/LAPP server.

\item  The CMS/LAPP server receives the AMQP request message from the
  \texttt{"subcontractor.service"} queue:
  \vskip 10pt
  \begin{center}
    \scalebox{0.6}{\input{\pdftextimgpath/vire_cmslapp_rpc_payload_7.pdftex_t}}
  \end{center}

\item  The CMS/LAPP server decodes/deserializes the AMQP request message
  using the  \texttt{"content\_type"} message   property.
  \vskip 10pt
  \begin{center}
    \scalebox{0.6}{\input{\pdftextimgpath/vire_cmslapp_rpc_payload_8.pdftex_t}}
  \end{center}

\item  The CMS/LAPP server analyzes the type of payload request object
  and perform a first check to determine if it is possible to accomplish
  the action. If some error is detected at this stage,
  a special failure \emph{response} payload object is built. It the RPC request payload seems to be
  possible to fulfill/execute, the CMS/LAPP server performs the necessary task(s) to
  accomplish the request. If an error is detected during this step, a special failure \emph{response} payload
  object  is built. Else, a specific success \emph{response} payload object is built.
  Whatever type of payload (success/failure) is built, it is embedded in a response message.
  The message is then serialized using the Protobuf  coding and wrapped in an AMQP response message:

  \vskip 10pt
  \begin{center}
    \scalebox{0.6}{\input{\pdftextimgpath/vire_cmslapp_rpc_payload_9.pdftex_t}}
  \end{center}

  The AMQP response message is given the proper \texttt{"correlation\_id"} and \texttt{"content\_type"}
  properties which are associated to the serialized buffer.

\item  The CMS/LAPP server sends the AMQP response message to the \texttt{"reply\_to"} queue,
  as requested by the Vire server.

  \vskip 10pt
  \begin{center}
    \scalebox{0.5}{\input{\pdftextimgpath/vire_cmslapp_rpc_payload_10.pdftex_t}}
  \end{center}

\item The  RPC client plug of  the Vire server reads  the AMQP response
  message from the \\
  \texttt{"vireserver.service.response.uFaezae2"} queue.

  \vskip 10pt
  \begin{center}
    \scalebox{0.5}{\input{\pdftextimgpath/vire_cmslapp_rpc_payload_11.pdftex_t}}
  \end{center}

\item The  RPC client plug decodes/deserializes the AMQP response
  message using the \texttt{"content\_type"} message property.

  \vskip 10pt
  \begin{center}
    \scalebox{0.5}{\input{\pdftextimgpath/vire_cmslapp_rpc_payload_12.pdftex_t}}
  \end{center}

\item  RPC client plug checks the type of payload response  object
  and transmits it to the Vire server.

  \vskip 10pt
  \begin{center}
    \scalebox{0.5}{\input{\pdftextimgpath/vire_cmslapp_rpc_payload_13.pdftex_t}}
  \end{center}

\end{enumerate}






\vfill
\afterpage{\clearpage}
\pagebreak

%%%%%%%%%%%%%%%%%%%%%%%%%%%%%%%%%%%%%%%%%%%%%%%%%%%%%%%%%%%%%%%%%%%%%%%%%%%%%%%%%%
\subsubsection{System event queues between the Vire server and the CMS/LAPP server}

THIS SECTION IS EMPTY. IT WILL BE DETAILED LATER.

\vfill
\afterpage{\clearpage}
\pagebreak
