\subsection{Communication domains accessible to the CMS/LAPP server}

The CMS/LAPP server and the  Vire software services (Vire server, Vire
clients) can communicate together through communication channels using
the  AMQP   protocol.   A  RabbitMQ   server  is  used  as   the  core
communication system.

By   convention,   communication    channels   are   confined   within
\emph{communication domains} (implemented through \emph{virtual hosts}
in terms of the  AMQP system).  A domain lies in  one of the following
categories:

\begin{itemize}

\item
  Category    \textcolor{blue}{\texttt{"vire::com::domain::system"}}:
  conventionally reserved for private communication at system/sysadmin
  level,

\item
  Category \textcolor{green}{\texttt{"vire::com::domain::monitoring"}}
  :  conventionally reserved  for  communications  at data  monitoring
  level (data read polling, pub/sub events\dots),

\item
  Category   \textcolor{red}{\texttt{"vire::com::domain::control"}}  :
  conventionally  reserved for  communications at  data control  level
  (configuration, priviledged/write operations on hardware\dots).

\end{itemize}

For the Vire-CMS/LAPP interface, there are two available domains:

\begin{itemize}

\item   Domain/vhost    \texttt{"/supernemo/demonstrator/cms/vire/subcontractors/cmslapp/system"}   (of
  category   \textcolor{blue}{\texttt{"vire::com::domain::system"}}):
  used for system RPC transactions and asynchronous messaging (events,
  alarms, signals)  between the Vire  server and the  CMS/LAPP server.
  No Vire client are allowed to use this domain.

\item    Domain/vhost   \texttt{"/supernemo/demonstrator/cms/vire/monitoring"}    (of
  category\\
  \textcolor{green}{\texttt{"vire::com::domain::monitoring"}})  : used
  for data polling RPC transactions and asynchronous alarm, log and Pub/Sub event messages.
  Vire  clients, the  Vire server  and the  CMS/LAPP server  are given
  access to this domain.

  %% \item       the       \texttt{vire\_cms\_control}       (category
  %%   \textcolor{blue}{\texttt{vire::com::domain::monitoring})     is
  %%   used for data polling RPC transactions and asynchronous Pub/Sub
  %%   messages.   Vire  clients, the  Vire  server  and the  CMS/LAPP
  %%   server may be given access to this domain.

\end{itemize}

Within  a given  thematic domain,  various communication  channels
are implemented by default:

\begin{itemize}

\item for the \texttt{"/supernemo/demonstrator/cms/vire/subcontractors/cmslapp/system"} domain,
  only private queues are used for RPC or event messenging. Their names are fixed by design.

\item for the \texttt{"/supernemo/demonstrator/cms/vire/monitoring"} domain,
  only public exchanges are used for RPC or event messenging. Their names are fixed by design.

\end{itemize}

%% synchronous RPC  point-to-point bindings, asynchronous event
%% queues,      asynchronous      Pub/Sub     exchanges\dots\      Figure
%% \ref{fig:cmslapp_domains:layout_0}   shows    the   layout    of   the
%% communication domains accessible by the Vire and CMS/LAPP servers with
%% possible associated communication channels.

%% \begin{figure}[h]
%% \begin{center}
%% \scalebox{0.9}{\input{\pdftextimgpath/vire_cmslapp_domains-layout_0.pdftex_t}}
%% \end{center}
%% \caption{The communication  domains between  the CMS/LAPP  server, the
%%   Vire            server            and            the            Vire
%%   client(s).}\label{fig:cmslapp_domains:layout_0}
%% \end{figure}

To gain  access to  a given  domain, a software  component must  use a
dedicated user  account (RabbitMQ  user) with proper  privileges.  The
RabbitMQ  server   is  thus  conventionally  preconfigured   with  the
following setup:

\begin{itemize}

%% \item   The    Vire   server   uses   the    RabbitMQ   user   account
%%   \texttt{"viresvr"}         to         access        both         the\\
%%   \texttt{"/supernemo/demonstrator/cms/vire/subcontractors/cmslapp/system"}
%%   and \texttt{"/supernemo/demonstrator/cms/vire/monitoring"} virtual hosts.

\item The CMS/LAPP server uses the RabbitMQ user \texttt{"cmslappsvr"}
  to  access  both   \\
  \texttt{"/supernemo/demonstrator/cms/vire/subcontractors/cmslapp/system"}  and
  \texttt{"/supernemo/demonstrator/cms/vire/monitoring"} virtual hosts.

\end{itemize}

\noindent  \textbf{Note:}   There  is  no  official   predefined  user
account(s) for SuperNEMO Vire clients.  The Vire server is responsible
to create and then destroy on  the fly temporary RabbitMQ accounts for
specific  client  sessions.  A  dedicated  \texttt{RabbitMQManager}
service is embedded in the Vire server to perform such operations. The
CMS/LAPP server is not involved in these administrative operations.


\vfill
\afterpage{\clearpage}
\pagebreak
