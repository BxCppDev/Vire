
%%%%%%%%%%%%%%%%%%%%%%%%%%%%%%%%%%%%%%%%%%%%%%%%%%%%%%%%%%%%%%%%%%%%%%
\subsection{Configuration parameters for the CMS/LAPP server}

\subsubsection{Configuration file of the CMS/LAPP server}

Specific setup  files are used  to store the  configuration parameters
used  (and  possibly shared)  by  the  Vire  server and  the  CMS/LAPP
subcontractor server.   These files must  be available from  a central
configuration directory\footnote{In the example, we use an environment
  variable to define this base directory.}  and passed to the Vire and
CMS/LAPP servers at startup.

These files, as well as other configuration files, will be distributed
and shared through a VCS  \footnote{Version control system.  Note that
  the  SuperNEMO experiment  uses a  Subversion server  hosted at  LPC
  \url{https://nemo.lpc-caen.in2p3.fr/svn}}.  It  is ensured  that the
configuration files  loaded at startup  by both the Vire  and CMS/LAPP
servers  will be  synchronized through  the VCS,  even if  they run  on
different  nodes.    Appendix  \ref{app:filesystem_conf}   proposes  a
technique to identify the paths of all configuration files.
As the files should contains some credential informations, the VCS should
be private.


\vskip 10pt
\noindent \textbf{Example (the names of the executable are arbitrary)}:
\begin{itemize}

\item to run the CMS/LAPP server, we should typically use:
  \vskip 10pt
  \small
  \begin{Verbatim}[frame=single,xleftmargin=0.cm,label=\fbox{shell}]
    shell$ cmslappserver \
    --setup-config=${SNEMO_ONLINE_CONFIG_BASE_DIR}/snemo/cms/1.0.2/subcontractors/cmslapp/setup.conf \
    ...
  \end{Verbatim}
  \normalsize

\item to run the Vire server, we should typically use:
  \vskip 10pt
  \small
  \begin{Verbatim}[frame=single,xleftmargin=0.cm,label=\fbox{shell}]
    shell$ vireserver \
    --setup-config=${SNEMO_ONLINE_CONFIG_BASE_DIR}/snemo/cms/1.0.2/vireserver/setup.conf \
    ...
  \end{Verbatim}
  \normalsize
\end{itemize}

\noindent      In      the      example      above,      the      file
\verb|snemo/cms/1.0.2/subcontractors/cmslapp/setup.conf| is distributed  through a private VCS in
a conventional  base directory  accessible by  both Vire  and CMS/LAPP
servers. The  environment variable \verb|SNEMO_ONLINE_CONFIG_BASE_DIR|
is  used  to  locate  this  base  directory.  It  is  defined  in  the
environment from which the CMS/LAPP runs (i.e. a Bash shell). Example:
\vskip 10pt
\small
\begin{Verbatim}[frame=single,xleftmargin=0.cm,label=\fbox{shell}]
shell$ export SNEMO_ONLINE_CONFIG_BASE_DIR="/home/snemo/config"
\end{Verbatim}
\normalsize

\noindent\textbf{Note:} as credentials will be probably stored in some startup files
(RabbitMQ \emph{user/password} pairs), extreme care should be taken while distributing the configuration
files. This has to be discussed.


\subsubsection{Configuration parameters}

The CMS/LAPP server  must know a minimal set of  parameters that allow
to identify the experimental setup and connect to the Vire system with
a non ambiguous configuration.

A private configuration  file is used to store fundamental parameters about
the experimental setup.

The configuration  file should  store the  following set  of mandatory
parameters:

\begin{itemize}

\item  The   \texttt{setup\_name}  character  string  is   the  unique
  identifier label of the experimental setup. \\
  \noindent \textbf{Example:} \verb+"snemo"+.

\item The \texttt{setup\_version} character string encodes the version
  number  of  the  experimental  setup.  It  should  typically  use  a
  conventional         sequence-based        identifier         format
  (\emph{major.minor.revision} scheme).\\
  \noindent \textbf{Example:} \verb+"1.0.2"+ or \verb+"test_42"+

\item  The \texttt{setup\_description}  character string  contains the
  short description of the experimental setup. \\
  \noindent \textbf{Example:} \verb+"The SuperNEMO Demonstrator Experiment"+

\item  The \texttt{amqp\_svr\_host}  character string  encodes the  IP
  address  or   the  URL  of   the  main  RabbitMQ  server   used  for
  communication between remote components involved in the system.\\
  \noindent \textbf{Example:} \texttt{"sndemo.in2p3.fr"} or \texttt{192.10.0.23}

\item The \texttt{amqp\_svr\_port} integer is  the port number used by
  the main RabbitMQ server.\\
  \noindent \textbf{Example:} \texttt{5672}

\item   The  \texttt{amqp\_cmslapp\_user}   character  string
  stores  the user login used by the CMS/LAPP server to enter
  the RabbitMQ system. % (use the \textcolor{blue}{service} virtual host).
  \noindent \textbf{Example:} \texttt{"cmslappserver"}

\item   The  \texttt{amqp\_cmslapp\_password}   character  string
  stores  the user credentials used by the CMS/LAPP server to enter
  the RabbitMQ system.\\
  \noindent \textbf{Example:} \texttt{"Sésame, ouvre-toi!"}\\
  \noindent \textbf{Note:} for security reasons, this parameter could be stored or transmit to the server
  by another mechanism (to be discussed).

  %% CMS/LAPP-Vire server system domain:

\item   The  \texttt{amqp\_cmslapp\_system\_vhost}   character  string
  stores  the  identifier  of the  \textcolor{blue}{service}  RabbitMQ
  virtual host used to confine  the system private communications with
  the Vire server.\\
  \noindent \textbf{Example:} \texttt{"/supernemo/demonstrator/cms/vire/subcontractors/cmslapp/system"}

\item      The     \texttt{amqp\_cmslapp\_system\_service\_queue}
  character string stores the name of  the RabbitMQ queue used to read
  RPC request messages sent by the Vire server.\\
  \noindent \textbf{Example:} \texttt{"subcontractor.service"}

\item      The     \texttt{amqp\_cmslapp\_system\_event\_queue}
  character string stores the name of  the RabbitMQ queue used to read
  event messages sent by the Vire server.\\
  \noindent \textbf{Example:} \texttt{"subcontractor.event"}

\item      The     \texttt{amqp\_vireserver\_system\_event\_queue}
  character string stores the name of  the RabbitMQ queue used to send
  event/signal messages to the Vire server.\\
  \noindent \textbf{Example:} \texttt{"vireserver.event"}

\item      The     \texttt{amqp\_vireserver\_system\_service\_queue}
  character string stores the name of  the RabbitMQ queue used to send
  RPC request messages messages to the Vire server.\\
  \noindent \textbf{Example:} \texttt{"vireserver.service"}

  %% Monitoring domain:
\item   The  \texttt{amqp\_cms\_monitoring\_vhost}   character  string
  stores the identifier  of the \textcolor{green}{monitoring} RabbitMQ
  virtual host  used to confine the  monitoring communications between
  the Vire server, the Vire clients and the CMS/LAPP server.\\
  \noindent \textbf{Example:} \texttt{"/supernemo/demonstrator/cms/vire/monitoring"}

\item      The     \texttt{amqp\_cms\_monitoring\_resource\_service\_exchange}
  character string stores the name of  the RabbitMQ exchange to be subscribed to read
  RPC request messages sent by the Vire server or Vire clients.\\
  \noindent \textbf{Example:} \texttt{"resource\_request.service"}

\item      The     \texttt{amqp\_cms\_monitoring\_alarm\_event\_exchange}
  character string stores the name of  the RabbitMQ exchange where to send
  alarm event messages to be read by the Vire server or Vire clients.\\
  \noindent \textbf{Example:} \texttt{"alarm.event"}

\item      The     \texttt{amqp\_cms\_monitoring\_log\_event\_exchange}
  character string stores the name of  the RabbitMQ exchange where to send
  log event messages to be read by the Vire server or Vire clients.\\
  \noindent \textbf{Example:} \texttt{"log.event"}

\item      The     \texttt{amqp\_cms\_monitoring\_pubsub\_event\_exchange}
  character string stores the name of  the RabbitMQ exchange where to send
  Pub/Sub event messages to be read by the Vire server or Vire clients.\\
  \noindent \textbf{Example:} \texttt{"log.event"}


  %% Device naming/resolution:

\item   The  \texttt{mos\_device\_launching\_path}   character  string
  represents  the path  of  the configuration  file  which stores  the
  informations  about  the devices  managed  by  MOS servers  and  the
  mapping rules to address them in  the Vire server naming scheme. \\
  \noindent \textbf{Example:} \texttt{"\${SNEMO\_ONLINE\_CONFIG\_BASE\_DIR}/snemo/cms/1.0.2/devices\_launch.conf"}

  This  file stores  the informations  that describe  how OPCUA  based
  device  handlers are  instantiated within  MOS servers  and how  the
  OPCUA  devices  naming scheme  is  mapped  (mounted) into  the  Vire
  devices/resources naming scheme. An example  of such a file is shown
  on figure \ref{fig:cmslapp_server:dev_launch_conf}.

\end{itemize}

\vskip 10pt
\noindent An example of configuration  file for the CMS/LAPP server is
shown in Fig. \ref{fig:setup_conf}.

\begin{figure}[h]
\small
\begin{Verbatim}[frame=single,xleftmargin=0.cm,label=\fbox{Configuration}]
# Setup identification and description:
setup_name           = "snemo"
setup_version        = "1.0.2"
setup_description    = "The SuperNEMO Demonstrator Experiment"

# Access to the AMQP/RabbitMQ system and dedicated resources:
amqp_svr_host         = "cmssupernemo.in2p3.fr"
amqp_svr_port         = 5672
amqp_cmslapp_user     = "cmslappsvr"
amqp_cmslapp_password = "Sésame, ouvre-toi!"

# Vire server/CMSLAPP system virtual host;
amqp_cmslapp_system_vhost = "/supernemo/demonstrator/cms/vire/subcontractors/cmslapp/system"
amqp_cmslapp_system_service_queue    = "subcontractor.service"
amqp_cmslapp_system_event_queue      = "subcontractor.event"
amqp_vireserver_system_service_queue = "vireserver.service"
amqp_vireserver_system_event_queue   = "vireserver.event"

# Vire monitoring virtual host:
amqp_cms_monitoring_vhost = "/supernemo/demonstrator/cms/vire/monitoring"
amqp_cms_monitoring_resource_service_exchange = "resource_request.service"
amqp_cms_monitoring_alarm_event_exchange      = "alarm.event"
amqp_cms_monitoring_log_event_exchange        = "log.event"
amqp_cms_monitoring_pubsub_event_exchange     = "pubsub.event"

# Description of devices managed by MOS servers
# and mounted by the Vire server through the CMS/LAPP proxy server:
mos_device_launching_path   = \
  "${SNEMO_ONLINE_CONFIG_BASE_DIR}/snemo/cms/1.0.2/devices_launch.conf"
...
\end{Verbatim}
\normalsize
\caption{Example of configuration file for  the CMS/LAPP server with a
  set      of      parameters      shared      with      the      Vire
  server.}\label{fig:setup_conf}
\end{figure}


\begin{figure}[p]
  \begin{center}
    \rotatebox{90}{\framebox{\BVerbatimInput[
          label=\fbox{\texttt{devices_launch.conf}},
          fontsize=\tiny,
          firstline=1,
          lastline=58]
        {cms_server/samples/devices_launch.conf}}}
  \end{center}
  \caption{Example of \texttt{devices\_launch.conf} file.}
  \label{fig:cmslapp_server:dev_launch_conf}
\end{figure}


\vfill
\afterpage{\clearpage}
\pagebreak
