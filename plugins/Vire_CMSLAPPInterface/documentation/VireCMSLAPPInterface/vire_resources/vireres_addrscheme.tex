

\subsection{Vire/CMS addressing scheme}

The mapping of any Vire resource  in terms of MOS addressing scheme is
resolved  by the  CMS/LAPP server  through  the rules  defined in  the
\verb|devices_launch.conf|     file    (see     example    on     Fig.
\ref{fig:cmslapp_server:dev_launch_conf}).   The  CMS/LAPP  server  is
responsible to build a local resource map that associates any resource
path  to  an effective  item  (datapoint  read/write action,  methods)
accessible from a MOS.

\subsubsection{Example : resources from a FSM-like device}

The Vire server defines the  \verb|SuperNEMO:/Demonstrator/CMS/Acquisition/start| resource.
From this unique Vire path, the CMS server must extract the
corresponding \emph{parent path} and \emph{base name}.

\begin{itemize}
\item Parent Vire path:  \verb|SuperNEMO:/Demonstrator/CMS/Acquisition/|
\item Base name:  \verb|start|
\end{itemize}

Then the CMS/LAPP server resolves the MOS address of the parent path using the informations
stored in the \verb|devices_launch.conf| file:
\begin{itemize}
\item MOS server: \texttt{192.168.1.15}
\item MOS port: \texttt{48040}
\item Device namespace: \texttt{CMS}
\item Root device name: \texttt{DAQ}
\end{itemize}

Now the device is non ambiguously located in the OPCUA space.

The resource  base name (here \verb|start|)  is thus identified/mapped
as  the  \texttt{start} method  of  the  DAQ  device model.   This  is
guaranted by  the XML to  Vire configuration files  convertion utility
(available  from the  \texttt{Vire\_MOS} software  plugin).  The  same
procedure         can         be        applied         to         the\\
\verb|SuperNEMO:/Demonstrator/CMS/Acquisition/stop| resource.

\subsubsection{Example : resources from a datapoint hosted in a device}


The            Vire            server           defines            the
\verb|SuperNEMO:/Demonstrator/CMS/Coil/PS/Monitoring/Voltage/__dp_read__|
resource associated  to the \texttt{.../Voltage}  datapoint monitoring
functionality.

The CMS/LAPP server must extract the corresponding  \emph{parent path} and \emph{base name}.

\begin{itemize}
\item Parent Vire path:  \verb|SuperNEMO:/Demonstrator/CMS/Coil/PS/Monitoring/Voltage/|
\item Base name:  \verb|__dp_read__|
\end{itemize}


Then the CMS/LAPP server resolves the MOS address of the parent path using the informations
stored in the \verb|devices_launch.conf| file:
\begin{itemize}
\item MOS server: \texttt{192.168.1.15}
\item MOS port: \texttt{4841}
\item Device namespace: \texttt{CMS}
\item Root device name: \texttt{COIL\_PS.Monitoring.Voltage}
\end{itemize}

The  Vire  mount  point  is  extracted  from  the  Vire  parent  path:
\verb|SuperNEMO:/Demonstrator/CMS/Coil/PS/|.  This allows to identify
the top level device handled by the MOS server (\texttt{CMS.COIL\_PS})
then find the sub-path  to the datapoint \texttt{Monitoring.Voltage}.
Now the datapoint  is non ambiguously located in the  OPCUA space. The
resource base name (here \verb|__dp_read__|)  is identified as the
conventional  operation that  reads the  current cached  value of  the
datapoint.

\subsubsection{CMS/LAPP map of resources}

As soon  as the CMS/LAPP server  has accepted the connection  with the Vire
server,  and following  the rules  explained above,  it's possible  to
build a map of all resources requested by the Vire server.  The map is
built from  the list of  requested resources  and the contents  of the
\texttt{devices\_launch.conf} file.

Once the  connection with the  Vire server has been  established, each
time the CMS/LAPP  server will receive a request message  about a resource,
it will immediately know, thanks to this map, what device or datapoint
is targeted and which operations are possible.
Table \ref{tab:cms:resource_map}  displays an excerpt of
such a map.

\begin{sidewaystable}[p]
\begin{center}%
\footnotesize
\begin{tabular}{|l|c|l|l|l|}
\hline
\textbf{Vire resource}  &  \textbf{M/C} & \textbf{OPCUA address \& name} & \textbf{Type of} & \textbf{Operation} \\
 &   & \ & \textbf{ object} &  \\
\hline
\hline
\verb|SuperNEMO:/Demonstrator/CMS/Acquisition/start| & C &
\texttt{192.168.1.15:48040:CMS.DAQ} & Device & \texttt{start} method \\
\hline
\verb|SuperNEMO:/Demonstrator/CMS/Acquisition/stop| & C &
\texttt{192.168.1.15:48040:CMS.DAQ} & Device & \texttt{stop} method \\
\hline
\verb|SuperNEMO:/Demonstrator/CMS/Coil/PS/Monitoring/Voltage/__dp_read__| & M &
\texttt{192.168.1.15:4841:CMS.COIL\_PS.Monitoring.Voltage} & Datapoint & read the value \\
\hline
\verb|SuperNEMO:/Demonstrator/CMS/Coil/PS/Monitoring/Current/__dp_read__| &M &
\texttt{192.168.1.15:4841:CMS.COIL\_PS.Monitoring.Current} & Datapoint & read the value \\
\hline
\verb|SuperNEMO:/Demonstrator/CMS/Coil/PS/Control/Voltage/__dp_read__| &M &
\texttt{192.168.1.15:4841:CMS.COIL\_PS.Control.Voltage} & Datapoint & read the value \\
\hline
\verb|SuperNEMO:/Demonstrator/CMS/Coil/PS/Control/Voltage/__dp_write__| & C &
\texttt{192.168.1.15:4841:CMS.COIL\_PS.Control.Voltage} & Datapoint & write the value \\
\hline
\verb|SuperNEMO:/Demonstrator/CMS/Coil/PS/Control/Current/__dp_read__| &M &
\texttt{192.168.1.15:4841:CMS.COIL\_PS.Control.Current} & Datapoint & read the value \\
\hline
\verb|SuperNEMO:/Demonstrator/CMS/Coil/PS/Control/Current/__dp_write__| & C &
\texttt{192.168.1.15:4841:CMS.COIL\_PS.Control.Current} & Datapoint & write the value \\
\hline
\end{tabular}
\normalsize
\end{center}
\caption{Example of a map of resources with their correspondances with OPCUA datapoints read/write access
and devices methods.}
\label{tab:cms:resource_map}
\end{sidewaystable}
