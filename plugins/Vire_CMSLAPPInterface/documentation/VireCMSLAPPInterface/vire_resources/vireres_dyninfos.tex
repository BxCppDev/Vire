
\subsection{Dynamic informations associated to the resources}
\label{sec:vire_resources:dynamic_infos}

Due to  the realtime and hardware  nature of the devices  that publish
them to  the Vire system,  each resource  is assigned some  flags that
reflect  its current  status.  Four \emph{resource  status flags}  are
defined for each resource:

\begin{itemize}

\item the \emph{missing} flag is set  when the resource is not present
  or is not accessible from the CMS/LAPP server.

  Typically, this occurs  when the device that  publishes the resource
  is not  accessible because of  a temporary disconnection of  the MOS
  server  or a  communication error  between  the MOS  server and  the
  device.   It  is   obvious  that  the  disconnection   of  a  device
  automatically sets the missing flag  for all resources associated to
  this deviced.  It is  the responsibility of  the CMS/LAPP  server to
  detect such  event and  update the missing  flags associated  to the
  relevant resources. Additional event/alarm will also be sent through
  dedicated communication channels.

\noindent\textbf{Example:}     Assume    the     MOS    running     at
\texttt{192.168.1.15:4841}   and  hosting   the  \texttt{CMS.COIL\_PS}
device is disconnected  because our colleague Naughty  Zoot teared off
the cable  connecting the coil  power supply device to  the computer
hosting the  MOS server.   Then all resources  identified with  a path
starting  with \verb|"SuperNEMO:/Demonstrator/CMS/Coil/PS/"|  will be
automatically tagged as \emph{missing} by the CMS/LAPP server.

\item the \emph{disabled}  flag is set when the  resource is temporary
  not \emph{executable}  because of the current  context.  This occurs
  for example when a finite state machine is in some specific internal
  state, preventing some of the transitions of its transition table to
  be invoked.

\noindent\textbf{Example:} Assume the DAQ  device, which implements an
internal finite state machine, is  in the \emph{running} state.  It is
expected in this case that the \texttt{start} transition method cannot
be used, whereas the \texttt{stop} transition method can be invoked at
any  time.   From the  point  of  view  of  Vire resources:\\  --  the
\verb|SuperNEMO:/Demonstrator/CMS/Acquisition/start| resource is thus
\emph{disabled}           \\            --           and           the
\verb|SuperNEMO:/Demonstrator/CMS/Acquisition/stop|    resource    is
\emph{enabled}.  As  soon as the internal  state of the DAQ  daemon is
changed  to  \emph{stopped} thanks  to  a  call to  the  \texttt{stop}
transition,  the \verb|SuperNEMO:/Demonstrator/CMS/Acquisition/start|
resource       is      \emph{enabled}       again      while       the
\verb|SuperNEMO:/Demonstrator/CMS/Acquisition/stop|    resource    is
tagged with the \emph{disabled} flag.

\item the  \emph{pending} flag  is set  when the  resource \textbf{is}
  executing.  This occurs when the execution of the resource cannot be
  considered  as  instantaneous and  implies  some  operations with  a
  significant latency  time.  The  CMS/LAPP server determines  that it
  will have  to wait  for a  while before the  full completion  of the
  resource   execution   and  the   making   of   a  proper   response
  (success+output parameters/failure+error record)  to the caller.  It
  thus sets the \emph{pending} flag associated to the resource as long
  as the processing is not terminated.

\noindent\textbf{Example:}  Assume   the  Vire  server   requests  the
execution                            of                            the
\\  \verb|SuperNEMO:/Demonstrator/CMS/Acquisition/start| resource  at
time $t_0$.  The  CMS server invokes the \texttt{start}  method of the
\texttt{192.168.1.15:48040:CMS.DAQ}  OPCUA  device.   Given  that  the
execution    of    this    method   lasts    several    seconds,    no
acknowledge/response  from  the  MOS  server  is  expected  before  an
arbitrary  time   $t_1>t_0$.   Thus  the  CMS/LAPP   server  sets  the
\emph{pending}                flag               on                the
\verb|SuperNEMO:/Demonstrator/CMS/Acquisition/start|  resource during
the time interval  $[t_0;t_1]$.  As soon as  the \texttt{start} method
is  proven terminated  (at  time $t_1$),  the  \emph{pending} flag  is
unset.

\item the  \emph{failed} flag is set  when a the device  associated to
  resource or a single resource itself met some internal error/failure
  state. Such an  error is detected at the discretion  of the CMS/LAPP
  server.

\noindent\textbf{Example:} none provided.


%% \noindent\textbf{Bad bad bad Example:}

%% The     Vire    server     requests    the     execution    of     the\\
%% \verb|SuperNEMO:/Demonstrator/CMS/Coil/PS/Control/Voltage/__dp_write__|
%% control resource with a value of the voltage which is out of the valid
%% range for this model of power supply device: $V$=5000 Volts > $V_{max}$=20
%% Volts.  Of  course, the MOS  server will reject  this request and  the CMS
%% server will  set the \emph{error} flag  and maintain it as  long as no
%% other  successful  execution  of  the resource  is  done.   This  flag
%% indicates that the last control operation on the resource has failed.

\end{itemize}


\noindent  Given these  four boolean  flags,  it is  possible for  the
CMS/LAPP  server to  build  a  4-bits status  word  that reflects  the
current  status of  any resource  at any  time. Accessing  this status
bitset for each resource, any client  of the CMS/LAPP server (the Vire
server application  or some Vire  client applications) can  figure out
the real time  conditions to access resources and build its own local
representation of the resources.

\begin{figure}[h]
\begin{center}
\scalebox{0.75}{\input{\pdftextimgpath/vire_resource-status_flags_0.pdftex_t}}
\end{center}
\caption{An example of a toy GUI  component which displays the status bits
  associated  to  a  resource,  informing the  user  of  the  realtime
  conditions. Here  the \textbf{M}issing  flag is not  set, reflecting
  that the resource  is available to the system  typically because the
  associated device is recognized by  the MOS and is currently running
  with all its functionalities.  The \textbf{D}isabled flag is not set
  which  means the  \texttt{\_\_dp\_write\_\_} resource  associated to
  this  datapoint can  be  used to  set the  voltage  set point.   The
  \textbf{P}ending  flag  is  set (\textcolor{red}{red}  light)  which
  means that a write operation,  previously initiated by some user, is
  currently   processing    and   not   terminated.     Finally,   the
  \textbf{F}ailed flag is not set, reflecting that no specific problem
  has been detected so far by  the CMS/LAPP server with this datapoint
  resource.}
\label{fig:vire_resource:status_flags_0}
\end{figure}

\vfill
\clearpage
\pagebreak
