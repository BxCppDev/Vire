
\subsection{Introduction}

The \emph{resource} is a central concept  in the Vire API.  A resource
is unique and identified with an unique \emph{path} in the Vire naming
scheme.

Each resource is associated to a  single operation that can be performed
from  some  unique component  (a  \emph{device})  of the  experimental
setup: it may consist in reading the current value of a datapoint (DP)
for  monitoring  purpose,  write  a  configuration  value  in  another
datapoint (control  operation), invoke  some transition method  from a
device implementing a finite state machine (FSM, \texttt{DAQstart}, \texttt{DAQstop} \dots).

Each      resource     can/must      be     classified      in     the
\textcolor{green}{\emph{monitoring}}                                or
\textcolor{red}{\emph{control}} category.

\noindent\textbf{Examples:}
\begin{itemize}
\item The resource identified with path: \\
  \verb|"SuperNEMO:/Demonstrator/CMS/Coil/PS/Control/Current/__dp_read__"|\\
  is  conventionally  classified  in  the  \textcolor{green}{\emph{monitoring}}  category
  because its role  is to fetch/read the current value  of a datapoint
  accessible    through   a    specific   MOS    server.   Here    the
  \texttt{.../Control/Current}  identifier  corresponds to  a  control
  datapoint but  this resource only \textcolor{green}{reads}  the value
  of the set point. It may be the cached last value recorded in a MOS server
  or the actual current value stored in the device itself.

  The \verb+__dp_read__+ name, as a  suffix of the full resource path,
  is a reserved keyword in the Vire API.

\item The resource identified with path: \\
  \verb|"SuperNEMO:/Demonstrator/CMS/Coil/PS/Control/Current/__dp_write__"|\\
  is conventionally classified in  the \textcolor{red}{\emph{control}} category because
  its role is to set/write the value of a datapoint accessible through
  a specific MOS server. Here again  the resource is associated to the
  \texttt{.../Control/Current}  datapoint;  however   it  is  used  to
  \textcolor{red}{write} the requested value of the set point. Such resource
  generally implies the user has specific privilege.

  The \verb+__dp_write__+ name, as a suffix of the full resource path,
  is a reserved keyword in the Vire API.

\item The resource identified with path: \\
  \verb|"SuperNEMO:/Demonstrator/CMS/Acquisition/stop"|\\
  is also  classified in the  \textcolor{red}{\emph{control}} category
  because it corresponds to  a method which \textcolor{red}{performs a
    transition} on a  specific finite state machine  (FSM) embedded in
  the MOS DAQ device.
\end{itemize}

Depending of the  size and compounding of  the experimental apparatus,
the  Vire  server  may  publish  a large  set  of  resources  for  its
clients. Some  of these  resources (but not  all) are  implemented and
accessible through  the CMS/LAPP server  because the devices  they are
related to are driven by some  MOS servers.  We thus need an interface
to make  these \emph{MOS embedded} resources  available and accessible
in the Vire system.

A  crucial point  is to  make  sure that  any resource  known by  Vire
through its path can be identified properly in the OPCUA device naming
scheme.   This has  been  evoked  in the  previous  section where  the
\texttt{devices\_launch.conf}     file     was     introduced     (see
Fig. \ref{fig:cmslapp_server:dev_launch_conf}).

\subsubsection{Example : resources published by a FSM-like device}

The SuperNEMO DAQ daemon/automaton (implemented  as a OPCUA device) is
managed through a  dedicated MOS server.  The DAQ  device is typically addressed
by the following set of parameters:

\begin{itemize}
\item MOS server: \texttt{192.168.1.15}
\item MOS port: \texttt{48040}
\item Device namespace: \texttt{CMS}
\item Root device name: \texttt{DAQ} (from the \texttt{CMS} root in the OPCUA namespace)
\end{itemize}

\noindent We assume that two  methods are publicly available from this
device.   These  methods   are   supposed  to   be   defined  in   the
ICD\footnote{Interface  Control  Document}   and  the  associated  XML
description of the device (for example: \\
\verb|${SNEMO_ONLINE_CONFIG_BASE_DIR}/snemo/1.0.2/cms/MOS/SNEMO/DAQ/DAQdevice_1.xml|):
\begin{itemize}

\item the \texttt{CMS.DAQ.start} method starts the data acquisition (FSM transition)

\item the \texttt{CMS.DAQ.stop} method stops the data acquisition (FSM transition)

\end{itemize}

\noindent In this special case, Vire considers that each of these methods is a \emph{resource}
which can be executed.
Mapping the DAQ daemon object in the OPCUA space as the
\verb|SuperNEMO:/Demonstrator/CMS/Acquisition/| path in Vire's resource space, we can thus define
two resources in Vire:
\begin{itemize}
\item     the    \verb|SuperNEMO:/Demonstrator/CMS/Acquisition/start|
  \textcolor{red}{control   resource}   is   associated  to   \\   the
  \texttt{192.168.1.15:48040:CMS.DAQ.start} method,
\item     the     \verb|SuperNEMO:/Demonstrator/CMS/Acquisition/stop|
  \textcolor{red}{control   resource}   is   associated  to   \\   the
  \texttt{192.168.1.15:48040:CMS.DAQ.stop} method.
\end{itemize}

\noindent Vire      behaves      like      a      filesystem      where      the
\texttt{192.168.1.15:48040:CMS.DAQ}  OPCUA device  is \emph{mounted}
at the Vire  \verb|SuperNEMO:/Demonstrator/CMS/Acquisition| mount point. From
this mount point,  one may navigate and access to  various devices and
datapoints.


\subsubsection{Example : resources from datapoints hosted in a device}

\noindent The  SuperNEMO coil power  supply  device  is managed  through  a dedicated  MOS
server. This device is addressed by the following set of parameters:

\begin{itemize}
\item MOS server: \texttt{192.168.1.15}
\item MOS port: \texttt{4841}
\item Device namespace: \texttt{CMS}
\item Root device name: \texttt{COIL\_PS}
\end{itemize}

\noindent According    to     the    description    of    the     device  \\
  (the \verb|${SNEMO_ONLINE_CONFIG_BASE_DIR}/snemo/1.0.2/cms/MOS/SNEMO/COIL/COIL_PS_1.xml|     file    defined     in    the
\verb|${SNEMO_ONLINE_CONFIG_BASE_DIR}/snemo/1.0.2/cms/devices_launch.conf|  file), the  following four  datapoints are
published in the OPCUA space:

\begin{itemize}
\item      \texttt{Monitoring.Current}      with      read      access
  (\textcolor{green}{\emph{monitoring}} only)
\item      \texttt{Monitoring.Voltage}      with      read      access
  (\textcolor{green}{\emph{monitoring}} only)
\item   \texttt{Control.Current}   with    read   and   write   access
  (\textcolor{green}{\emph{monitoring}}                            and
  \textcolor{red}{\emph{control}})
\item   \texttt{Control.Voltage}   with    read   and   write   access
  (\textcolor{green}{\emph{monitoring}}                            and
  \textcolor{red}{\emph{control}})
\end{itemize}

\noindent From the Vire point of view, this implies the existence of six resources with the following paths:
\begin{itemize}
\item \verb|SuperNEMO:/Demonstrator/CMS/Coil/PS/Monitoring/Current/__dp_read__|
\item \verb|SuperNEMO:/Demonstrator/CMS/Coil/PS/Monitoring/Voltage/__dp_read__|
\item \verb|SuperNEMO:/Demonstrator/CMS/Coil/PS/Control/Current/__dp_read__|
\item \verb|SuperNEMO:/Demonstrator/CMS/Coil/PS/Control/Voltage/__dp_read__|
\item \verb|SuperNEMO:/Demonstrator/CMS/Coil/PS/Control/Current/__dp_write__|
\item \verb|SuperNEMO:/Demonstrator/CMS/Coil/PS/Control/Voltage/__dp_write__|
\end{itemize}

Here   the  \verb|__dp_read__|   and  \verb|__dp_write__|   names  are
conventionally defined in the Vire API.  They corresponds respectively
to \emph{read}  and \emph{write}  methods (accessors) associated  to a
datapoint.   It  is  the  responsibility of  the  CMS/LAPP  server  to
translate  them in  term of  MOS specific  actions, respectively  read
(\texttt{get})  and   write  (\texttt{set})  the  value   of  a  given
datapoint.

Vire mounts  the \texttt{192.168.1.15:4841:CMS.COIL\_PS}  OPCUA device
at  the  Vire   \\  \verb|SuperNEMO:/Demonstrator/CMS/Coil/PS|  mount
point.
