
\subsection{Actions on resources}
\label{sec:vire_resources:actions}

THIS SECTION NEEDS REWRITING...

We  consider now  the  typical case  where the  Vire  server (or  Vire
clients connected to the CMS/LAPP server) access the devices and datapoints
in polling mode.   Basically, the Vire server can  request two kinds
of RPC action on any resource managed by the CMS/LAPP server:

\begin{itemize}
\item executing the resource (in \emph{blocking} mode or \emph{non blocking} mode),
\item fetching the status of the resource.
\end{itemize}

\subsubsection{Execute a resource} %% in \emph{blocking} mode:}

The execution  of a resource is  equivalent to the invocation  of some
specific   device  method   or  datapoint   accessor  (get/set   OPCUA
method).  The Vire  server  (or  client) is  expected  to request  the
execution      of      a       resource      with      a      specific
\texttt{vire::cms::resource\_exec\_request} payload object.

The CMS/LAPP server is expected to send back its response using
a \texttt{vire::cms::resource\_exec\_failure\_response} or a
 \texttt{vire::cms::resource\_exec\_success\_response}.

%% \begin{verbatim}

%% struct resource_exec_request
%% {
%%   std::string path; // Identifier of the resource
%%   std::vector<argument_type> input_arguments; // The list of input arguments.
%%                                               // The exact definition of this list
%%                                               // depends on the nature of the
%%                                               // resource.
%% }
%% \end{verbatim}


The Vire server waits for the response of the  CMS/LAPP server until the  operation has
been completed or failed.

Question: support for timeout.


Typical JSON  formatted \emph{resource execution request}  objects are
shown        on         figures        \ref{cms:json:resource_exec_0},
\ref{cms:json:resource_exec_1} and \ref{cms:json:resource_exec_2}.

  \begin{figure}[h]
  \footnotesize
  \begin{Verbatim}[frame=single,xleftmargin=0.cm,label=\fbox{\texttt{JSON}}]
{
  "path" : "SuperNEMO://Demonstrator/CMS/Acquisition/start",
  "input_args" : [ ]
}
\end{Verbatim}
  \normalsize
  \caption{Typical JSON formatted  \emph{resource  execution request}  object
    sent by  the Vire server to  the CMS server. Here  the resource
    corresponds    to   a    method    that    takes   no    input
    arguments.}\label{cms:json:resource_exec_0}
  \end{figure}


  \begin{figure}[h]
  \footnotesize
  \begin{Verbatim}[frame=single,xleftmargin=0.cm,label=\fbox{\texttt{JSON}}]
{
    "path" : "SuperNEMO://Demonstrator/CMS/Coil/PS/Control/Current/__dp_write__",
    "input_args" : [
      { "setpoint" : "0.341" }
    ]
}
\end{Verbatim}
  \normalsize
  \caption{Typical JSON formatted   \emph{resource  execution request}  object
    sent by  the Vire server to  the CMS server. Here  the resource
    corresponds    to  a write    method    that    takes  only one    input
    argument.}\label{cms:json:resource_exec_1}
  \end{figure}

  \begin{figure}[h]
  \footnotesize
  \begin{Verbatim}[frame=single,xleftmargin=0.cm,label=\fbox{\texttt{JSON}}]
{
    "path" : "SuperNEMO://Demonstrator/CMS/Coil/PS/Control/Current/__dp_read__",
    "input_args" : []
}
\end{Verbatim}
  \normalsize
  \caption{Typical  JSON formatted \emph{resource  execution request} object
    sent by  the Vire server to  the CMS server. Here  the resource
    corresponds    to a read    method    that    takes  no input
    argument.}\label{cms:json:resource_exec_2}
  \end{figure}

The CMS/LAPP  server must decode a  message which wraps the  request object
and,  if  it  makes  sense,  launch the  execution  of  the  requested
operation in  the targeted MOS. When  it receives the answer  from the
MOS,  it is  supposed to  build  and send  a response  message to  the
caller.

In case of successful execution, a
 \texttt{resource\_exec\_success\_response} object is built by the CMS server and
delivered back to the
Vire server.


%% \begin{verbatim}
%% struct resource_exec_success_response
%% {
%%   // Nested type alias:
%%   typedef vire::resource::resource_status_record resource_status_record;

%%   resource_status_record status;
%%   std::vector<argument> ouput_arguments;
%% };
%% \end{verbatim}

A  typical JSON  formatted \emph{resource\_exec\_success\_response} object is
shown    on    Fig.     \ref{cms:json:resource_exec_response_0}    and
\ref{cms:json:resource_exec_response_1}.

  \begin{figure}[h]
  \footnotesize
  \begin{Verbatim}[frame=single,xleftmargin=0.cm,label=\fbox{\texttt{JSON}}]
{
  "status" : {
    "path" : "SuperNEMO://Demonstrator/CMS/Acquisition/start",
    "flags" : "0100"
  },
  "reception_timestamp" : "20160203T175203.031743801", // TO BE CHECKED
  "completion_timestamp" : "20160203T175217.56143801", // TO BE CHECKED
  "output_arguments" : []
}
\end{Verbatim}
  \normalsize
  \caption{A  JSON formatted \emph{resource  execution success  response}
    object sent  by the  CMS server  to the  Vire server. Here
    the response does not contain  any output arguments. Note that the
    CMS  server  also informs  the  Vire  server about  the updated
    resource status.
    %the  reception
    %timestamp  of  the  resource  execution message  as  well  as  the
    %timestamp                       at                       execution
    %completion/failure.
  }
  \label{cms:json:resource_exec_success_response_0}
  \end{figure}

  \begin{figure}[h]
  \footnotesize
  \begin{Verbatim}[frame=single,xleftmargin=0.cm,label=\fbox{\texttt{JSON}}]
{
  "status" : {
    "path" : "SuperNEMO://Demonstrator/CMS/Coil/PS/Control/Current/__dp_read__",
    "flags" : "0000",
  }
  "reception_timestamp" : "20160203T175203.031743801",
  "completion_timestamp" : "20160203T175217.56143801",
  "output_arguments" : [
    "value" : "1.23"
  ]
}
\end{Verbatim}
  \normalsize
  \caption{JSON formatted \texttt{resource\_exec\_success\_response} object.}
  \label{cms:json:resource_exec_success_response_1}
  \end{figure}

%\vskip10pt


In case of failure, the CMS server builds a failure response object which
describes the error (figure  \ref{cms:json:resource_exec_response_2}).

  \begin{figure}[h]
  \footnotesize
  \begin{Verbatim}[frame=single,xleftmargin=0.cm,label=\fbox{\texttt{JSON}}]
{
   "error_type_id" : {
      "name" : "invalid_status_error",
      "version" : "1.0"
   },
   "error" : {
      "code" : "101"
      "message" : "Operation failed because the DAQ device is already running."
      "path" : "SuperNEMO://Demonstrator/CMS/DAQ/start",
      "flag" : "disabled"
    }
  }
}
\end{Verbatim}
  \normalsize
  \caption{Example of  JSON formatted \emph{resource  execution failure response} object
    emitted by  the CMS server to  the Vire server. This is typical case where
    the execution of the resource has failed because the resource was not available
    when invoked.
  }
  \label{cms:json:resource_exec_response_2}
  \end{figure}

List of error object types embedded in the  \texttt{resource\_exec\_failure\_response}:

\begin{itemize}

\item  \texttt{invalid\_context\_error} :  the general context does
  not allow to operate the execution of a resource.

\item \texttt{invalid\_resource\_error} : the resource path is not valid.

\item \texttt{invalid\_status\_error} : the resource status is bad.

\item  \texttt{resource\_exec\_error}  :  the  execution  of  the
  resource met an error: invalid context, invalid input argument name,
  invalid input argument value, timeout\dots

\end{itemize}

\vskip 5mm
  \par\textcolor{red}{NOTE (FM): Management of timeout:
    Should the Vire server add a timeout info in the resource\_exec message
    to set a limit to the response waiting time ?  if the cms is not able to answer
    within the timeout, an error response message is sent back to the Vire server.
  }

\clearpage
\pagebreak
% XxX
\subsubsection{Execute a resource in \emph{non blocking/asynchronous} mode:}

A resource  can be  executed in \emph{non  blocking/asynchronous} mode  still using
a \texttt{resource\_exec\_request} payload  object and a synchronous RPC
communication system.
The content of the payload request object is the same than in \emph{blocking} mode.
The only difference is in the header of the message where a specific
\texttt{asynchronous} flag is set, with additional dedicated metadata
to specify some message queue where to sent the response message.

At reception of the request payload object, the  CMS/LAPP server checks
the \texttt{asynchronous} flag and fetch the metadata from the message header
which informs of the message queue where the response will be later transmitted as an event
payload. The CMS/LAPP server then immediately sents back a conventionnal response object
to the \texttt{"reply\_to"} RPC queue.

TODO: DESCRIPTION OF THE RESPONSE :
\texttt{resource\_exec\_async\_success\_response}
\texttt{resource\_exec\_async\_failure\_response}

The \texttt{"pending"} flag associated to the resource
is set until the  operation has been completed or failed.

%% \begin{itemize}

%% \item \texttt{resource\_exec\_non\_blocking\_success\_response} object:

%%   \begin{figure}[h]
%%   \footnotesize
%%   \begin{Verbatim}[frame=single,xleftmargin=0.cm,label=\fbox{\texttt{JSON}}]
%% {
%%   "status" : {
%%     "path" : "SuperNEMO://Demonstrator/CMS/Coil/PS/Control/Current/__dp_read__",
%%     "flags" : "0010",
%%   }
%% }
%% \end{Verbatim}
%%   \normalsize
%%   \caption{JSON formatted \texttt{resource\_exec\_non\_blocking\_success\_response} object.}
%%   \label{cms:json:resource_exec_non_blocking_success_response}
%%   \end{figure}

%% \item \texttt{resource\_exec\_non\_blocking\_failure\_response} object:

%%  TO BE DONE.

%% \end{itemize}

The pending flag is set until the  operation has been completed or failed.

\clearpage
\pagebreak

\subsubsection{Fetch the status bitset of a resource:}

At  any time,  the Vire  server can  ask the  CMS server  to send  the
current status  of a resource.  Typical  JSON formatted \emph{resource
  fetch    status}    request    messages   are    shown    on    Fig.
\ref{cms:json:resource_fetch_status_0}                             and
\ref{cms:json:resource_fetch_status_1} (for the response message).

RPC payload types:
\begin{itemize}
\item request: \texttt{"vire::cms::resource\_fetch\_status\_request"}
\item success response: \texttt{"vire::cms::resource\_fetch\_status\_success\_response"}
\item failure response: \texttt{"vire::cms::resource\_fetch\_status\_failure\_response"}
\end{itemize}

Attributes types embedded in response payload:
\begin{itemize}
\item \texttt{"vire::cms::resource\_status\_record"}
\item \texttt{"vire::utility::invalid\_context\_error"}
\item \texttt{"vire::cms::invalid\_resource\_error"}
\end{itemize}

%% TODO: consider errors...
%%This action never fails for it is always possible for the CMS server to build such an answer.


  \begin{figure}[h]
  \footnotesize
  \begin{Verbatim}[frame=single,xleftmargin=0.cm,label=\fbox{\texttt{JSON}}]
{
  "path" : "SuperNEMO://Demonstrator/CMS/Coil/PS/Control/Current/__dp_read__"
}
\end{Verbatim}
  \normalsize
  \caption{JSON formatted  \emph{resource  fetch status}  object
    sent by  the Vire server to  the CMS server.}\label{cms:json:resource_fetch_status_0}
  \end{figure}

  \begin{figure}[h]
  \footnotesize
  \begin{Verbatim}[frame=single,xleftmargin=0.cm,label=\fbox{\texttt{JSON}}]
{
  "status" : {
    "path" : "SuperNEMO://Demonstrator/CMS/Coil/PS/Control/Current/__dp_read__",
    "timestamp" : "2016-02-03 17:52:03.031743801+01:00",
    "flags" : "0010"
  }
}
\end{Verbatim}
%    "timestamp" : "2016-02-03 17:52:03.031743801+01:00",
  \normalsize
  \caption{JSON formatted \emph{resource  fetch status}  success response message
    sent back by  the CMS server to  the Vire server.}\label{cms:json:resource_fetch_status_1}
  \end{figure}

\clearpage
\pagebreak

\subsubsection{Activate pub/sub of a resource:}

The Vire server is able to request the dynamic activation/deactivation
of the Pub/Sub  service associated to a given  resource. Three payload
objects are thus  defined to address this kind  of transaction between
the Vire and CS servers:

\begin{itemize}

\item \texttt{vire::resource::resource\_pubsub\_request}  wraps a Vire
  server  request to  activate/deactivate  the Pub/Sub  service for  a
  given resource managed through the CMS server.

\item     \texttt{vire::resource::resource\_pubsub\_success\_response}
  wraps a successful response of the CMS server.

\item     \texttt{vire::resource::resource\_pubsub\_failure\_response}
  wraps a failure response of the CMS server.

\end{itemize}

Specific utility objects are also defined:
\begin{itemize}

\item \texttt{vire::resource::amqp\_pubsub\_access} : describes the access to the Pub/Sub service
for a given resource.

\end{itemize}

These objects are described in appendix \ref{app:payload}.
